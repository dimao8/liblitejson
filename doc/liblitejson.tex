\documentclass[a4paper, draft]{article}

\usepackage[russian]{babel}
\usepackage[utf8]{inputenc}
\usepackage[T2A]{fontenc}
\usepackage{indentfirst}
\usepackage{hyperref}
\usepackage{graphicx}

\title{Библиотека LiteJSON \\ Руководство пользователя \\ Ревизия 1.0}
\author{Хрущев Дмитрий aka DimaO}
\date{2022}

\begin{document}
\maketitle
\newpage
\tableofcontents
\newpage

\section{О проекте LiteJSON}
Проект LiteJSON разрабатывается как легковесная класс-ориентированная Библиотека
работы с файлами с разметкой JSON. Интерфейс библиотеки представлен классом
json_loader и содержит базовые операции извлечения и добавления данных в файл
(или массив в памяти) с разметкой JSON.
\end{document}
