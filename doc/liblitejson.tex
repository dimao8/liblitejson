\documentclass[a4paper, a4paper]{article}

\usepackage[utf8]{inputenc}
\usepackage[T2A]{fontenc}
\usepackage[russian]{babel}
\usepackage{indentfirst}
\usepackage{hyperref}
\usepackage{graphicx}
\usepackage{longtable}
\usepackage{amsmath}

\title{Библиотека LiteJSON \\ Руководство пользователя \\ Ревизия 1.0}
\author{Хрущев Дмитрий aka DimaO}
\date{2022}

\begin{document}
\maketitle
\newpage
\tableofcontents
\newpage

\section{О проекте LiteJSON}
Проект LiteJSON разрабатывается как легковесная класс-ориентированная Библиотека
работы с файлами с разметкой JSON. Интерфейс библиотеки представлен классом
\texttt{json\_loader} и содержит базовые операции извлечения и добавления данных в файл
(или массив в памяти) с разметкой JSON.

\section{БНФ файла JSON}

\begin{verbatim}
<value> ::= <object> | <array> | <number> | <string>
  | "true" | "false" | "null"
\end{verbatim}

\begin{verbatim}
<object> ::= "{" <entry list> | <empty> "}"
\end{verbatim}

\begin{verbatim}
<entry list> ::= <entry pair> | <entry list> "," <entry pair>
\end{verbatim}

\begin{verbatim}
<entry pair> ::= <string> ":" <value>
\end{verbatim}

\begin{verbatim}
<array> ::= "[" <value list> | <empty> "]"
\end{verbatim}

\begin{verbatim}
<value list> ::= <value> | <value list> "," <value>
\end{verbatim}

После получения списка токенов из файла выстраиваем дерево. Дерево разбора
выстраивается рекурсивно. Корнем всегда является объект, следовательно после
разбора дерева список должен схлопнуться в единственный объект.

Приведем пример разбора с тремя вложенными объектами.

\begin{verbatim}
{
  "object1" : {
    "object2" : {
      "object3" : 1,
      "object4" : "text"
    }
  }
}
\end{verbatim}

\begin{enumerate}
  \item Начнем разбор с извлечения первого, корневого \texttt{<value>}. Для
        определения типа \texttt{<value>} вычитываем первый токен. Он может
        быть только <<\texttt{\{}>>, <<\texttt{[}>>, <string>, <number>,
        <<\texttt{true}>>, <<\texttt{false}>> или <<\texttt{null}>>.
  \item Определив тип следующего токена, начинаем его разбор.
        \begin{enumerate}
          \item <<\texttt{\{}>> обозначает начало объекта. Внутри объекта список
                из пар \texttt{<string> : <value>}. Список можут быть пустой,
                или может содержать одну или несколько пар. Пары разделяются
                запятой. Извлекаем строку имени ключа, двоеточие, затем повторяем
                пункт~1. После извлечения значения определяем, есть ли запятая.
                Если она есть, извлекаем ее. Затем выполняем извлечение еще раз.
                Не забываем извлечь закрывающую <<\texttt{\}}>>.
          \item <<\texttt{[}>> обозначает начало списка. Список может быть пустым,
                или содержать одно или несколько \texttt{<value>}. Разбираем
                \texttt{<value>} по пункту~1. Далее, если есть запятая, продолжаем
                извлекать \texttt{value}. После окончания извлечения всего списка
                извлекаем закрывающую <<\texttt{]}>>.
          \item \texttt{<string>} --- обычная строка. Начинается с символа \verb!"!
                и заканчивается символом \verb!"!.
          \item \texttt{<number>} --- любое число. JSON не делит числа на
                плавающие и целые. Число начинается либо с <<->>, либо с нуля,
                либо с другой последовательности цифр. Целое число не может
                начинаться с нуля, если не равно нулю. Далее может следовать
                дробная часть, представляющая собой целое число, отделенное от
                предыдущей части точкой. Затем может следовать экспонента.
                Экспонента начинается с символа <<E>> или <<e>> после чего
                может идти знак <<+>> или <<-->>. Затем идет целое число.
          \item \texttt{true} --- просто литерал \texttt{true}.
          \item \texttt{false} --- просто литерал \texttt{false}.
          \item \texttt{null} --- просто литерал \texttt{null}.
        \end{enumerate}
    \end{enumerate}

Каждый раз, когда внутри списка или объекта мы встречаем другое \texttt{<value>},
мы начинаем его разбор рекурсивно по тому же правилу. \texttt{value} формируются
динамически и добавляются в объекты и списки более высокого уровня до тех пор,
пока не наступит выход из рекурсии, или не закончится список токенов.

\end{document}
